%% Copyright 2009 Ivan Griffin
%
% This work may be distributed and/or modified under the
% conditions of the LaTeX Project Public License, either version 1.3
% of this license or (at your option) any later version.
% The latest version of this license is in
%   http://www.latex-project.org/lppl.txt
% and version 1.3 or later is part of all distributions of LaTeX
% version 2005/12/01 or later.
%
% This work has the LPPL maintenance status `maintained'.
% 
% The Current Maintainer of this work is Ivan Griffin
%
% This work consists of the files periodic_table.tex

%Description
%-----------
%periodic_table.tex - an example file illustrating the Periodic
%                     Table of Chemical Elements using TikZ

%Created 2009-12-08 by Ivan Griffin.  Last updated: 2010-01-11
%
%Thanks to Jerome
%-------------------------------------------------------------

\documentclass[12pt]{article}

\usepackage[utf8]{inputenc}
\usepackage{amsmath}
%%%<
\usepackage{verbatim}
%%%>

\begin{comment}
:Title: Periodic Table of Chemical Elements

\end{comment}

\usepackage{ifpdf}
\usepackage{tikz}
\usepackage[active,tightpage]{preview}
\usetikzlibrary{shapes,calc}

\ifpdf
  %
\else
  % Implement Outline text using pstricks if regular LaTeX->dvi->ps->pdf route
  \usepackage{pst-all}
\fi

\begin{document}

\newcommand{\CommonElementTextFormat}[5]
{
  \begin{minipage}{3.5cm}
    \centering
      {\ #3\hfill #1}%
      \linebreak
      {\ \hfill #2}
       \linebreak \linebreak
      {\textbf{#4}}%
      \linebreak \linebreak
      {{#5}}
  \end{minipage}
}

\newcommand{\EmptyNode}
{
  \begin{minipage}{3.3cm}
   
  \end{minipage}
}


\newcommand{\NaturalElementTextFormat}[5]
{
  \CommonElementTextFormat{#1}{#2}{#3}{\LARGE {#4}}{#5}
}

\newcommand{\OutlineText}[1]
{
\ifpdf
  % Couldn't find a nicer way of doing an outline font with TikZ
  % other than using pdfliteral 1 Tr
  %
  \pdfliteral direct {0.5 w 1 Tr}{#1}%
  \pdfliteral direct {1 w 0 Tr}%
\else
  % pstricks can do this with \pscharpath from pstricks
  %
  \pscharpath[shadow=false,
    fillstyle=solid,
    fillcolor=white,
    linestyle=solid,
    linecolor=black,
    linewidth=.2pt]{#1} 
\fi
}

\newcommand{\SyntheticElementTextFormat}[4]
{
\ifpdf
  \CommonElementTextFormat{#1}{#2}{\OutlineText{\LARGE #3}}{#4}
\else
  % pstricks approach results in slightly larger box
  % that doesn't break, so fudge here
  \CommonElementTextFormat{#1}{#2}{\OutlineText{\Large #3}}{#4}
\fi
}

\begin{preview}
\begin{tikzpicture}[font=\sffamily, scale=0.8, transform shape]

%% Fill Color Styles
   \tikzstyle{ElementFill} = [fill=yellow!5]
  \tikzstyle{QuarksFill} = [fill=yellow!35]
  \tikzstyle{BosonsFill} = [fill=orange!55]
  \tikzstyle{LeptonsFill} = [fill=green!40]
  \tikzstyle{NeutrinosFill} = [fill=blue!25]
%% Element Styles
  \tikzstyle{Element} = [draw=black, ElementFill,
    minimum width=4cm, minimum height=4cm, node distance=4cm]
  \tikzstyle{Quarks} = [Element, QuarksFill]
  \tikzstyle{Bosons} = [Element, BosonsFill]
  \tikzstyle{Leptons} = [Element, LeptonsFill]
  \tikzstyle{Neutrinos} = [Element,NeutrinosFill]
  \tikzstyle{Dummy} = [Element,ElementFill,scale=0.8]
  \tikzstyle{PeriodLabel} = [font={\sffamily\LARGE}, node distance=2.0cm]
  \tikzstyle{GroupLabel} = [font={\sffamily\LARGE}, minimum width=2.5cm, node distance=2.5cm]
  \tikzstyle{TitleLabel} = [font={\sffamily\Huge\bfseries}, node distance=7.0cm]
  \tikzstyle{Empty} = [  minimum width=4cm, minimum height=4cm, node distance=4cm  ]

%% quarks
  \node[name=U, Quarks] {\NaturalElementTextFormat{+2/3}{1,5-4 MeV}{1/2}{u}{up}};
   \node[name=D, below of=U, Quarks] {\NaturalElementTextFormat{-1/3}{4-8 MeV}{1/2}{d}{down}};

  \node[name=C, right of=U, Quarks] {\NaturalElementTextFormat{+2/3}{1,15-1,35 GeV}{1/2}{c}{charm}};
   \node[name=S, below of=C, Quarks] {\NaturalElementTextFormat{-1/3}{80-130 MeV}{1/2}{s}{strange}};

 \node[name=T, right of=C, Quarks] {\NaturalElementTextFormat{+2/3}{173,3$\pm$0,8 GeV}{1/2}{t}{top}};
   \node[name=B, below of=T, Quarks] {\NaturalElementTextFormat{-1/3}{4,1-4,4 GeV}{1/2}{b}{bottom}};
   
   \node[name=EU, below of=D, Empty]{\EmptyNode};
   
      \node[name=UD, left of=U, Empty]{\EmptyNode};
   
   %% leptons
  \node[name=E, below of=EU, Leptons] {\NaturalElementTextFormat{-1}{511 keV}{1/2}{e}{electron}};
   \node[name=M, right of=E, Leptons] {\NaturalElementTextFormat{-1}{105,66 MeV}{1/2}{$\boldsymbol{\mu}$}{muon}};
   \node[name=TA, right of=M, Leptons] {\NaturalElementTextFormat{-1}{1,777 GeV}{1/2}{$\boldsymbol{\tau}$}{tau}};
   
      %% neutrinos
  \node[name=NE, below of=E, Neutrinos] {\NaturalElementTextFormat{0}{$<$2,5 eV}{1/2}{$\boldsymbol{\nu_e}$}{neutrino e}};
   \node[name=NM, below of=M,  Neutrinos] {\NaturalElementTextFormat{0}{$<$170 keV}{1/2}{$\boldsymbol{\nu_\mu}$}{neutrino $\mu$}};
   \node[name=NT, below of=TA, Neutrinos] {\NaturalElementTextFormat{0}{$<$18 MeV}{1/2}{$\boldsymbol{\nu_\tau}$}{neutrino $\tau$}};

   \node[name=TY, right of=T, Empty]{\EmptyNode};
   
         %% bosons
  \node[name=Y, right of=TY, Bosons] {\NaturalElementTextFormat{0}{0}{0}{$\boldsymbol{\gamma}$}{photon}};

   \node[name=G, below of=Y, Bosons] {\NaturalElementTextFormat{0}{0}{1}{g}{gluon}};
   
       \node[name=H, below of=G, Bosons] {\NaturalElementTextFormat{0}{125,36$\pm$0,41 GeV}{0}{H}{Higgs}};
     

     \node[name=W, below of=H, Bosons] {\NaturalElementTextFormat{$\pm$1}{80,385$\pm$0,015 GeV}{1}{W}{Weak}};
     
     \node[name=W, below of=W, Bosons] {\NaturalElementTextFormat{0}{91,187$\pm$0,002 GeV}{1}{Z}{}};

%% Group
 \node[name=Quarks, above of=U, GroupLabel] {Quarks};
 \node[name=Leptons, above of=E, GroupLabel] {Leptons};
 \node[name=Neutrinos, below of=NE, GroupLabel] {Neutrinos};
 \node[name=Bosons, above of=Y, GroupLabel] {Bosons};
 
 \node[name=UT, above of=T,Empty]{\EmptyNode};
 
%
%%% Diagram Title
\node at (UT.north)[name=diagramTitle, TitleLabel]    {Particules élémentaires du modèle standard};

 \node at (diagramTitle.west -| UD.north) [name=D,Dummy] {\NaturalElementTextFormat{charge/e}{masse $  \cdot \ c^2$}{spin}{symbole}{nom}};



\end{tikzpicture}
\end{preview}
\end{document}